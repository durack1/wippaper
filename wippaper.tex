% git commit -am 'This is Karls comment about the commit'
% git push ;  This automatically updates taylor13 branch, and BECAUSE a PR is already queued, it updates the PR


% \documentclass[12pt,twocolumn]{article}
% Copernicus stuff
\documentclass[gmd,manuscript]{copernicus}
%\documentclass[gmd,manuscript]{../171128_Copernicus_LaTeX_Package/copernicus} 
% page/line labeling and referencing
% from http://seananderson.ca/2013/04/28/cross-referencing-reviewer-replies-in-latex.html

% \newcommand{\pllabel}[1]{\label{p-#1}\linelabel{l-#1}}
% \newcommand{\plref}[1]{see page~\pageref{p-#1}, line~\lineref{l-#1}.}
% answer environment for reviewer responses
% \newenvironment{answer}{\color{blue}}{}
% \usepackage{enumitem}

% \hypersetup{colorlinks=true,urlcolor=blue,citecolor=red}
% \hypersetup{colorlinks=false}
% \newcommand{\degree}{\ensuremath{^\circ}}
% \newcommand{\order}{\ensuremath{\mathcal{O}}}
% \newcommand{\bibref}[1] { \cite{ref:#1}}
% \newcommand{\pipref}[1] {\citep{ref:#1}}
% \newcommand{\ceqref}[1] {\mbox{CodeBlock \ref{code:#1}}}
% \newcommand{\charef}[1] {\mbox{Chapter \ref{cha:#1}}}
% \newcommand{\eqnref}[1] {\mbox{Eq.     \ref{eq:#1}}}
% \newcommand{\figref}[1] {\mbox{Figure   \ref{fig:#1}}}
% \newcommand{\secref}[1] {\mbox{Section  \ref{sec:#1}}}
% \newcommand{\appref}[1] {\mbox{Appendix \ref{sec:#1}}}
% \newcommand{\tabref}[1] {\mbox{Table   \ref{tab:#1}}}
% \newcommand{\urlref}[2] {\href{#1}{#2}\footnote{\url{#1}, retrieved \today.}}

% \newcommand{\editorial}[1]{\protect{\color{red}#1}}

\runningtitle{WIP Paper Draft \today}
\runningauthor{Balaji et al.}

\begin{document}

\title{Requirements for a global data infrastructure in support of CMIP6}

\Author[1,2]{Venkatramani}{Balaji}
\Author[3]{Karl E.}{Taylor}
\Author[4]{Martin}{Juckes}
\Author[5,4]{Bryan N.}{Lawrence}
\Author[3]{Paul J.}{Durack}
\Author[6]{Michael}{Lautenschlager}
\Author[7,2]{Chris}{Blanton}
\Author[8]{Luca}{Cinquini}
\Author[9]{S\'ebastien}{Denvil}
\Author[10]{Mark}{Elkington}
\Author[9]{Francesca}{Guglielmo}
\Author[9,4]{Eric}{Guilyardi}
\Author[4]{David}{Hassell}
\Author[11]{Slava}{Kharin}
\Author[6]{Stefan}{Kindermann}
\Author[1,2]{Sergey}{Nikonov}
\Author[7,2]{Aparna}{Radhakrishnan}
\Author[6]{Martina}{Stockhause}
\Author[6]{Tobias}{Weigel}
\Author[3]{Dean}{Williams}

\affil[1]{Princeton University, Cooperative Institute of Climate
  Science, Princeton, NJ 08540, USA}
\affil[2]{NOAA/Geophysical Fluid Dynamics Laboratory, Princeton, NJ 08540,
  USA}
\affil[3]{PCMDI, Lawrence Livermore National Laboratory, Livermore, CA 94550, USA}
\affil[4]{Science and Technology Facilities Council, Abingdon, UK}
\affil[5]{National Center for Atmospheric Science and University of
  Reading, UK}
\affil[6]{Deutsches KlimaRechenZentrum GmbH, Hamburg, Germany}
\affil[7]{Engility Inc., NJ, USA}
\affil[8]{Jet Propulsion Laboratory (JPL), 4800 Oak Grove Drive,
Pasadena, CA 91109, USA}
\affil[9]{Institut Pierre-Simon Laplace, CNRS/UPMC, Paris, France}
\affil[10]{Met Office, FitzRoy Road, Exeter, EX1 3PB, UK}
\affil[11]{Canadian Centre for Climate Modelling and Analysis, Atmospheric Environment Service, University of Victoria, BC, Canada}
% \affil[10]{NCAR}

\correspondence{V. Balaji (\texttt{balaji@princeton.edu})}

\received{}
\pubdiscuss{} %% only important for two-stage journals
\revised{}
\accepted{}
\published{}

%% These dates will be inserted by Copernicus Publications during the typesetting process.


\firstpage{1}

\maketitle

% \pagebreak
\abstract{The World Climate Research Programme (WCRP)'s Working Group
  on Climate Modelling (WGCM) Infrastructure Panel (WIP) was formed in
  2014 in response to the explosive growth in size and complexity of
  Coupled Model Intercomparison Projects (CMIPs) between CMIP3
  (2005-06) and CMIP5 (2011-12). This article presents the WIP
  recommendations for the global data infrastructure needed to support
  CMIP design, future growth and evolution. Developed in close
  coordination with those who build and run the existing
  infrastructure (the Earth System Grid Federation; ESGF), the
  recommendations are based on several principles beginning with the
  need to separate requirements, implementation and operations. Other
  important principles include the consideration of the diversity of
  community needs around data -- a \emph{data ecosystem} -- the
  importance of provenance, the need for automation, and the
  obligation to measure costs and benefits.
  
  This paper concentrates on requirements, recognising the diversity
  of communities involved (modelers, analysts, software developers,
  and downstream users). Such requirements include the need for
  scientific reproducibility and accountability alongside the need to
  record and track data usage. One key element is to generate a
  dataset-centric rather than system-centric focus, with an aim to
  making the infrastructure less prone to systemic failure.

  With these overarching principles and requirements, the WIP has
  produced a set of position papers, which are summarized in the
  latter pages of this document. They provide specifications for
  managing and delivering model output, including strategies for
  replication and versioning, licensing, data quality assurance,
  citation, long-term archival, and dataset tracking. They also
  describe a new and more formal approach for specifying what data,
  and associated metadata, should be saved, which enables future data
  volumes to be estimated, particularly for well-defined projects such
  as CMIP6.
 
  The paper concludes with a future-facing consideration of the global
  data infrastructure evolution that follows from the blurring of
  boundaries between climate and weather, and the changing nature of
  published scientific results in the digital age. }
% \pagebreak

\introduction
\label{sec:intro}

CMIP6 \citep{ref:eyringetal2016a}, the latest Coupled Model
Intercomparison Project (CMIP), can trace its genealogy back to the
Charney Report \citep{ref:charneyetal1979}. This seminal report on the
links between CO$_2$ and climate was an authoritative summary of the
state of the science at the time and produced findings that have stood
the test of time \citep{ref:bonyetal2013}. It is often noted
\citep[see, e.g][]{ref:andrewsetal2012} that the range and uncertainty
bounds on equilibrium climate sensitivity generated in this report
have not fundamentally changed, despite the enormous increase in
resources devoted to analysing the problem in decades since
\citep[see, e.g][]{ref:knuttietal2017}

Beyond its enduring findings on climate sensitivity, the Charney
Report also gave rise to a methodology for the treatment of
uncertainties and gaps in understanding, which has been equally
influential, and is in fact the basis of CMIP itself. The Report can
be seen as one of the first uses of the \emph{multi-model ensemble}.
At the time, there were two models available representing the
equilibrium response of the climate system to a change in CO$_2$
forcing, one from Syukuro Manabe's group at NOAA's Geophysical Fluid
Dynamics Laboratory (NOAA-GFDL) and the other from James Hansen's
group at NASA's Goddard Institute for Space Studies (NASA-GISS). Then
as now, these groups marshalled vast state-of-the-art computing and
data resources to run very challenging simulations of the Earth
system. The report's results were based on an ensemble of three runs
from the Manabe group, \citep[see e.g.][]{ref:manabewetherald1975} and
two from the Hansen group \citep[see e.g..][]{ref:hansenetal1981}.

The Atmospheric Model Intercomparison Project
\citep[AMIP:][]{ref:gates1992} was one of the first systematic
cross-model comparisons open to anyone who wished to participate. By
the time of the Intergovernmental Panel on Climate Change (IPCC)'s
First Assessment Report (FAR) in 1990 \citep{ref:houghtonetal1992},
the process had been formalized. At this stage, there were five models
participating in the exercise, and some of what is now called the
``Diagnosis, Evaluation, and Characterization of Klima'' \citep[DECK,
see][]{ref:eyringetal2016a} experiments\footnote{``Klima'' is German
  for ``climate''.} had been standardized (AMIP, a pre-industrial
control, 1\% per year CO$_2$ increase to doubling, etc). The future
``scenarios'' had emerged as well, for a total of five different
experimental protocols. Fast-forwarding to today,
\href{https://rawgit.com/WCRP-CMIP/CMIP6_CVs/master/src/CMIP6_source_id.html}{CMIP6
  expects more than 100
  models}\footnote{https://rawgit.com/WCRP-CMIP/CMIP6\_CVs/master/src/CMIP6\_source\_id.html,
  retrieved \today.} from
\href{https://rawgit.com/WCRP-CMIP/CMIP6_CVs/master/src/CMIP6_institution_id.html}{more
  than 40 modelling
  centres}\footnote{https://rawgit.com/WCRP-CMIP/CMIP6\_CVs/master/src/CMIP6\_institution\_id.html,
  retrieved \today.} \citep[in 27 countries, a stark contrast to the
US monopoly in][]{ref:charneyetal1979} to participate in the DECK and
historical experiments \citep[Table~2 of][]{ref:eyringetal2016a}, and
some subset of these to participate in one or more of the 23 MIPs
endorsed by the CMIP Panel \citep[Table~3 of][, originally 21 with two
new MIPs more recently endorsed]{ref:eyringetal2016a}. The
\href{https://rawgit.com/WCRP-CMIP/CMIP6_CVs/master/src/CMIP6_experiment_id.html}{MIPs
  call for 287
  experiments}\footnote{https://rawgit.com/WCRP-CMIP/CMIP6\_CVs/master/src/CMIP6\_experiment\_id.html,
  retrieved \today.} , a considerable expansion over CMIP5.

Alongside the experiments themselves is the
\href{http://clipc-services.ceda.ac.uk/dreq/index.html}{Data
  Request}\footnote{http://clipc-services.ceda.ac.uk/dreq/index.html,
  retrieved \today.} which defines, for each CMIP experiment, what
output each model should provide for analysis. The complexity of this
data request has also grown tremendously over the CMIP era. A typical
dataset from the FAR archive
(\href{https://cera-www.dkrz.de/WDCC/ui/cerasearch/entry?acronym=IPCC_DDC_FAR_GFDL_R15TRCT_D}{from
  the GFDL R15
  model}\footnote{https://cera-www.dkrz.de/WDCC/ui/cerasearch/entry?acronym=IPCC\_DDC\_FAR\_GFDL\_R15TRCT\_D,
  retrieved \today.} ) lists climatologies and time series of a few
basic climate variables such as surface air temperature, and the
dataset size is about 200~MB. The CMIP6 Data Request
\cite{ref:juckesetal2015} lists literally thousands of variables, from
8 modelling \emph{realms} (e.g. atmosphere, ocean, land, atmospheric
chemistry, land ice, ocean biogeochemistry and sea ice) from the
hundreds of experiments mentioned above. This growth in complexity is
testament to the modern understanding of many physical, chemical and
biological processes which were simply absent from the Charney
Report-era models.

The simulation output is now a primary scientific resource for
researchers the world over, rivaling the volume of observed weather
and climate data from the global array of sensors and satellites
\citep{ref:overpecketal2011}. Climate science, and observed and
simulated climate data in particular, have now become primary elements
in the ``vast machine'' \citep{ref:edwards2010} serving the global
climate and weather research enterprise.
% It could be worthwhile to quantify (in $USD) the impact, as forecasting
% in particular has yielded considerable social and economic gains

Managing and sharing this huge amount of data is an enterprise in its
own right -- and the solution established for CMIP5 was the global
Earth System Grid Federation
\citep[ESGF,][]{ref:williamsetal2011a,ref:williamsetal2015}. ESGF was
identified by the WCRP Joint Scientific Committee in 2013 as the
recommended infrastructure for data archiving and dissemination for
the Programme. A map of sites participating in the ESGF is shown in
Figure~\ref{fig:esgf} drawn from the
\href{https://portal.enes.org/data/is-enes-data-infrastructure/esgf}{IS-ENES
  Data
  Portal}\footnote{https://portal.enes.org/data/is-enes-data-infrastructure/esgf,
  retrieved \today.} . The sites are diverse and responsive to many
national and institutional missions. With multiple agencies and
institutions, and many uncoordinated and possibly conflicting
requirements, the ESGF itself is a complex and delicate artifact to
manage.

\begin{figure*}
  \begin{center}
    \includegraphics[width=175mm]{images/esgf-map-2017.png}
  \end{center}
  \caption{Sites participating in the Earth System Grid Federation in
    May 2017. Figure courtesy IS-ENES Data Portal. }
  \label{fig:esgf}
\end{figure*}

The sheer size and complexity of this infrastructure emerged as a
matter of great concern at the end of CMIP5, when the growth in data
volume relative to CMIP3 (from 40~TB to 2~PB, a 50-fold increase in 6
years) suggested the community was on an unsustainable path. These
concerns led to the 2014 recommendation of the WGCM to form an
\emph{infrastructure panel} (based upon
\href{https://drive.google.com/file/d/0B7Pi4aN9R3k3OHpIWC16Z0JBX3c/view?usp=sharing
}{a
  proposal}\footnote{https://drive.google.com/file/d/0B7Pi4aN9R3k3OHpIWC16Z0JBX3c/view?usp=sharing
  , retrieved \today.} at the 2013 annual meeting). The WGCM
Infrastructure Panel (WIP) was tasked with examining the global
computational and data infrastructure underpinning CMIP, and improving
communication between the teams overseeing the scientific and
experimental design of these globally coordinated experiments, and the
teams providing resources and designing that infrastructure. The
communication was intended to be two-way: providing input both to the
provisioning of infrastructure appropriate to the experimental design,
and informing the scientific design of the technical (and financial)
limits of that infrastructure.

This paper provides a summary of the findings by the WIP in the first
three years of activity since its formation in 2014, and the
consequent recommendations -- in the context of existing
organisational and funding constraints. In the text below, we refer to
\emph{findings}, \emph{requirements}, and \emph{recommendations}.
Findings refer to observations about the state of affairs:
technologies, resource constraints and the like, based upon our
analysis. Requirements are design goals that have been shared with
those building the infrastructure, such as the ESGF software and
security stack. Recommendations are our guidance to the community:
experiment designers, modelling centres, and the users of climate
data.

The intended audience for the paper is primarily the CMIP6 scientific
community. In particular, we aim to show how the scientific design of
CMIP6 as outlined in \cite{ref:eyringetal2016a} translates into
infrastructural requirements. We hope this will be instructive to the
MIP chairs and creators of multi-model experiments highlighting
resource implications of their experimental design, and for data
providers (modelling centres), to explain the sometimes opaque
requirements imposed upon them as a requisite for participation. By
describing how design of this infrastructure is severely constrained
by resources, we hope to provide a useful perspective to those who
find data acquisition and analysis a technical challenge. Finally, we
hope this will be of interest to general readers of the journal from
other geoscience fields, illuminating the particular character of
global data infrastructure for climate data, where the community of
users far outstrip in numbers and diversity, the Earth system
modelling community itself.

In Section~\ref{sec:principles}, the principles and scientific
rationale underlying the requirements for global data infrastructure
are articulated. In Section~\ref{sec:dreq} the CMIP6 Data Request is
covered: standards and conventions, requirements for modelling centres
to process a complex data request, and projections of data volume. In
Section~\ref{sec:licensing}, the recent evolution in how data are
archived is reviewed alongside a licensing strategy consistent with
current practice and scientific principle. In Section~\ref{sec:cite}
issues surrounding data as a citable resource are discussed, including
the technical infrastructure for the creation of citable data, and the
documentation and other standards required to make data a first-class
scientific entity. In Section~\ref{sec:replica} the implications of
data replicas, and in Section~\ref{sec:version} issues surrounding
data versioning, retraction, and errata are addressed.
Section~\ref{sec:summary} provides an outlook for the future of global
data infrastructure, looking beyond CMIP6 towards a unified view of
the ``vast machine'' for weather and climate data and computation.


\section{Principles and Constraints}
\label{sec:principles}

This section lays out some of the the principles and constraints which
have resulted from the evolution of infrastructure requirements since
the first CMIP experiment -- beginning with a historical context.

\subsection{Historical Context}
\label{sec:history}

In the pioneering days of CMIP, the community of participants was
small and well-knit, and all the issues involved in generating
datasets for common analysis from different modelling groups was
settled by mutual agreement (Ron Stouffer, personal communication).
Analysis was performed by the same community that performed the
simulations. The Program for Climate Model Diagnosis and
Intercomparison (PCMDI), established at Lawrence Livermore National
Laboratory (USA) in 1989, had championed the idea of more systematic
analysis of models, and in close cooperation with the climate
modelling centres, PCMDI assumed responsibility for much of the
day-to-day coordination of CMIP. Until CMIP3, the hosting of datasets
from different modelling groups could be managed at a single archival
site; PCMDI alone hosted the entire 40~TB archive.

From its earliest phases, CMIP grew in importance, and its results
have provided a major pillar that supports the periodic
Intergovernmental Panel on Climate Change (IPCC) assessment
activities. However, the explosive growth in the scope of CMIP,
especially between CMIP3 and CMIP5, represented a tipping point in the
supporting infrastructure. Not only was it clear that no one site
could manage all the data, the necessary infrastructure software and
operational principles could no longer be delivered and managed by
PCMDI alone.

For CMIP5, PCMDI sought help from a number of partners under the
auspices of the Global Organisation of Earth System Science Portals
(GO-ESSP). Many of the GO-ESSP partners who became the foundation
members and developers of the Earth System Grid Federation retargeted
existing research funding to help develop ESGF. The primary heritage
derived from the original U.S. Earth System Grid project funded by the
U.S. Department of Energy, but increasingly major contributions came
from new international partners. This meant that many aspects of the
ESGF system began from work which was designed in the context of
different requirements, collaborations and objectives. At the
beginning, none of the partners had funds for operational support for
the fledgling international federation, and even after the end of
CMIP5 proper (circa 2014), the ongoing ESGF has been sustained
primarily by small amounts of funding at a handful of the primary ESGF
sites. Most ESGF sites have had little or no formal operational
support. Many of the known limitations of the CMIP5 ESGF -- both in
terms of functionality and performance -- were a direct consequence of
this heritage.

With the advent of CMIP6 (in addition to some sister projects such as
obs4MIPs, input4MIPs and CREATE-IP), it was clear that a fundamental
reassessment would be needed to address the evolving scientific and
operational requirements. That clarity led to the establishment of the
WIP, but it has yet to lead to any formal joint funding arrangement --
the ESGF and the data nodes within it remain funded (if at all, many
data nodes are marginal activities supported on best efforts) by
national agencies with disparate timescales and objectives. Several
critical software elements also are being developed on volunteer
efforts and shoestring budgets. This finding has been noted in the US
National Academies Report on ``A National Strategy for Advancing
Climate Modeling'' \citep{ref:nasem2012}, which warned of the
consequences of inadequate infrastructure funding.

\subsection{Infrastructural Principles}
\label{sec:infra-principles}

\begin{enumerate}
\item With greater complexity and a globally distributed data
  resource, it has become clear that in the design of globally
  coordinated scientific experiments, the global computational and
  data infrastructure needs to be formally examined as an integrated
  element.
  
  The membership of the WIP, drawn as it is from experts in various
  aspects of the infrastructure, is a direct consequence of this
  requirement for integration. Representatives of modelling centres,
  infrastructure developers, and stakeholders in the scientific design
  of CMIP and its output comprise the panel membership. One of the
  WIP's first acts was to consider three phases in the process of
  infrastructure development: \emph{requirements},
  \emph{implementation}, and \emph{operations}, all informed by the
  builders of workflows at the modelling centres.
    
  \begin{itemize}
  \item The WIP, in consort with the WCRP's CMIP Panel, takes
    responsibility to articulate \emph{requirements} for the
    infrastructure.
  \item The \emph{implementation} is in the hands of the
    infrastructure developers, principally ESGF for the federated
    archive \citep{ref:williamsetal2015}, but also related projects
    like Earth System Documentation
    \citep[\href{https://www.earthsystemcog.org/projects/es-doc-models/
    }{ES-DOC}\footnote{https://www.earthsystemcog.org/projects/es-doc-models/
      , retrieved \today.} ,][]{ref:guilyardietal2013}.
  \item In 2016 at the WIP's request, the CMIP6 Data Node
    \emph{Operations} Team (CDNOT) was formed. It is charged with
    ensuring that all the infrastructure elements needed by CMIP6 are
    properly deployed and actually working as intended at the sites
    hosting CMIP6 data. It is also responsible for the operational
    aspects of the federation itself, including specifying what
    versions of the toolchain are run at every site at any given time,
    and organising coordinated version and security upgrades across
    the federation.
  \end{itemize}

  Although there is now a clear separation of concerns into
  requirements, implementation, and operations, close links are
  maintained by cross-membership between the key bodies, including the
  WIP itself, the CMIP Panel, the ESGF Executive Committee, and the
  CDNOT.
\item\label{broad} With the basic fact of anthropogenic climate change
  now well established \citep[see, e.g.,][]{ref:stockeretal2013} the
  scientific communities with an interest in CMIP is expanding. For
  example, a substantial body of work has begun to emerge to examine
  climate impacts. In addition to the specialists in Earth system
  science -- who also design and run the experiments and produce the
  model output -- those relying on CMIP output now include those
  developing and providing climate services, as well as
  \emph{consumers} from allied fields studying the impacts of climate
  change on health, agriculture, natural resources, human migration,
  and similar issues \citep{ref:mossetal2010}. This confronts us with
  a \emph{scientific scalability} issue (the data during its lifetime
  will be consumed by a community much larger, both in sheer numbers,
  and also in breadth of interest and perspective than the Earth
  system modelling community itself), which needs to be addressed.

  Accordingly, we note the requirement that infrastructure should
  ensure maximum transparency and usability for user (consumer)
  communities at some distance from the modelling (producer)
  communities.
\item\label{repro} While CMIP and the IPCC are formally independent,
  the CMIP archive is increasingly a reference in formulating climate
  policy. Hence the \emph{scientific reproducibility}
  \citep{ref:collinstabak2014} and the underlying \emph{durability}
  and \emph{provenance} of data have now become matters of central
  importance: being able to trace back, long after dataset creation,
  from model output to the configuration of models and the procedures
  and choices made along the way. This led the IPCC to require data
  distribution centres (DDCs) that attempt to guarantee the archival
  and dissemination of this data in perpetuity, and consequently to a
  requirement in the CMIP context of achieving reproducibility. Given
  the use of multi-model ensembles for both consensus estimates and
  uncertainty bounds on climate projections, it is important to
  document -- as precisely as possible, given the independent
  genealogy and structure of many models -- the details and
  differences among model configurations and analysis methods, to
  deliver both the requisite provenance and the routes to
  reproduction.
\item\label{analysis} With the expectation that CMIP DECK experiment
  results should be routinely contributed to CMIP, opportunities now
  exist for engaging in a more systematic and routine evaluation of
  Earth System Models (ESMs). This has led to community efforts to
  develop standard metrics of model ``quality''
  \citep{ref:eyringetal2016,ref:gleckleretal2016}. Typical multi-model
  analysis has hitherto taken the multi-model average, assigning equal
  weight to each model, as the most likely estimate of climate
  response. This ``model democracy'' \citep{ref:knutti2010} has been
  called into question and there is now a considerable literature
  exploring the potential of weighting models by quality
  \citep{ref:knuttietal2017}. The development of standard metrics
  would aid this kind of research.

  To that end, there is now a requirement to enable through the ESGF a
  framework for accommodating quasi-operational evaluation tools that
  could routinely execute a series of standardized evaluation tasks.
  This would provide data consumers with an increasingly (over time)
  systematic characterization of models. It may be some time before a
  fully operational system of this kind can be implemented, but
  planning must start now.

  In addition, there is an increased interest in climate analytics as
  a service \citep{ref:balajietal2011,ref:schnaseetal2017}. This
  follows the principle of placing analysis close to the data. Some
  centres plan to add resources that combine archival and analysis
  capabilities, e.g., NCAR's
  \href{https://www2.cisl.ucar.edu/resources/cmip-analysis-platform
  }{CMIP Analysis
    Platform}\footnote{https://www2.cisl.ucar.edu/resources/cmip-analysis-platform
    , retrieved \today.} , or the UK's JASMIN
  \citep{ref:lawrenceetal2013}.. There are also new efforts to bring
  climate data storage and analysis to the cloud era
  \citep[e.g][]{ref:duffyetal2015}. Platforms such as
  \href{http://pangeo-data.org/}{Pangeo}\footnote{http://pangeo-data.org/,
    retrieved \today.} show promise in this realm, and widespread
  experimentation and adoption is encouraged.
\item As the experimental design of CMIP has grown in complexity,
  costs both in time and money have become a matter of great concern,
  particularly for those designing, carrying out, and storing
  simulations. In order to justify commitment of resources to CMIP,
  mechanisms to identify costs and benefits in developing new models,
  performing CMIP simulations, and disseminating the model output need
  to be developed.

  To quantify the scientific impact of CMIP, measures are needed to
  \emph{track} the use of model output and its value to consumers. In
  addition to usage quantification, credit and tracing data usage in
  literature via citation of data is important. Current practice is at
  best citing large data collections provided by a CMIP participant,
  or all of CMIP. Accordingly, we note the need for a mechanism to
  identify and \emph{cite} data provided by each modelling centre.
  Alongside the intellectual contribution to model development, which
  can be recognized by citation, there is a material cost to centres
  in computing and data processing, which is both burdensome and
  poorly understood by those requesting, designing and using the
  results from CMIP experiments, who might not be in the business of
  model development. The criteria for endorsement introduced in CMIP6
  \citep[see Table~1 in][]{ref:eyringetal2016a} begins to grapple with
  this issue, but the costs still need to be measured and recorded. To
  begin documenting these costs for CMIP6, the ``Computational
  Performance'' MIP project (CPMIP) \citep{ref:balajietal2017} has
  been established, which will measure, among other things, throughput
  (simulated years per day) and cost (core-hours and joules per
  simulated year) as a function of model resolution and complexity.
  New tools for estimating data volumes have also been developed, see
  Section~\ref{sec:data-request} below.

\item\label{cmplx} Experimental specifications have become ever more
  complex, making it difficult to verify that experiment
  configurations conform to those specifications. Several modelling
  centres have encountered this problem in preparing for CMIP6,
  noting, for example, the challenging intricacies in dealing with
  input forcing data \citep[see][]{ref:duracketal2018}, output
  variable lists \citep{ref:juckesetal2015}, and crossover
  requirements between the endorsed MIPs and the DECK
  \citep{ref:eyringetal2016a} . Moreover, these protocols inevitably
  evolve over time, as errors are discovered or enhancements proposed,
  and centres needed to be adaptable in their workflows accordingly.
   
  We note therefore a requirement to encode the protocols to be
  directly ingested by workflows, in other words,
  \emph{machine-readable experiment design}. The intent is to avoid,
  as far as possible, errors in conformance to design requirements
  introduced by the need for humans to transcribe and implement the
  protocols, for instance, deciding what variables to save from what
  experiments. This is accomplished by encoding most of the
  specifications in standard, structured and machine readable text
  formats (XML and JSON) which can be directly read by the scripts
  running the model and post-processing, as explained further below in
  Section~\ref{sec:dreq}. The requirement spans all of the
  \emph{controlled vocabularies}
  (\href{https://github.com/WCRP-CMIP/CMIP6_CVs}{CMIP6\_CVs}\footnote{https://github.com/WCRP-CMIP/CMIP6\_CVs,
    retrieved \today.} : for instance the names assigned to models,
  experiments, and output variables) used in the CMIP protocols as
  well as the CMIP6 Data Request \citep{ref:juckesetal2015}, which
  must be stored in version-controlled, machine-readable formats.
  Precisely documenting the \emph{conformance} of experiments to the
  protocols \citep{ref:lawrenceetal2012} is an additional requirement.
  
\item\label{snap} The transition from a unitary archive at PCMDI in
  CMIP3 to a globally federated archive in CMIP5 led to many changes
  in the way users interact with the archive, which impacts management
  of information about users and complicates communications with them.
  In particular, a growing number of data users no longer registered
  or interacted directly with the ESGF. Rather they relied on
  secondary repositories, often copies of some portion of the ESGF
  archive created by others at a particular time (see for instance the
  \href{http://www.ipcc-data.org/docs/factsheets/TGICA_Fact_Sheet_CMIP5_data_provided_at_the_IPCC_DDC_Ver_1_2016.pdf
  }{IPCC CMIP5 Data
    Factsheet}\footnote{http://www.ipcc-data.org/docs/factsheets/TGICA\_Fact\_Sheet\_CMIP5\_data\_provided\_at\_the\_IPCC\_DDC\_Ver\_1\_2016.pdf
    , retrieved \today.} for a discussion of the snapshots and their
  coverage). This meant that reliance on the ESGF's inventory of
  registered users for any aspect of the infrastructure -- such as
  tracking usage, compliance with licensing requirements, or informing
  users about errata or retractions -- could at best ensure partial
  coverage of the user base.

  This key finding implies a more distributed design for several
  features outlined below, which devolve many of these features to the
  datasets themselves rather than the archives. One may think of this
  as a \emph{dataset-centric rather than system-centric} design (in
  software terms, a \emph{pull} rather than \emph{push} design):
  information is made available upon request at the user/dataset
  level, relieving the ESGF implementation of an impossible burden.
\end{enumerate}

Based upon the above considerations, the WIP produced a set of
position papers (see Appendix~\ref{sec:wip}) encapsulating
specifications and recommendations for CMIP6 and beyond. These papers,
summarised below, are available from the
\href{https://www.earthsystemcog.org/projects/wip/}{WIP
  website}\footnote{https://www.earthsystemcog.org/projects/wip/,
  retrieved \today.} . As the WIP continues to develop additional
recommendations, they too will be made available. As requirements
evolve, a modified document will be released with a new version
number.

\section{A structured approach to data production}
\label{sec:dreq}

The CMIP6 data framework has evolved considerably from CMIP5, and
follows the principles of scientific reproducibility (Item~\ref{repro}
in Section~\ref{sec:principles}) and the recognition that the
complexity of the experimental design (Item~\ref{cmplx}) required far
greater degrees of automation within the production workflow
generating simulation results. As a starting point, all elements in
the experiment specifications must be recorded in structured text
formats (XML and JSON, for example), and any changes must be tracked
through careful version control. \emph{Machine-readable} specification
of all aspects of the model output configuration is a design goal, as
noted earlier.

The data request spans several elements discussed in sub-sections
below.

\subsection{CMIP6 Data Request}
\label{sec:data-request}

The \href{http://clipc-services.ceda.ac.uk/dreq/index.html}{CMIP6 Data
  Request}\footnote{http://clipc-services.ceda.ac.uk/dreq/index.html,
  retrieved \today.} specifies which variables should be archived for
each experiment. It is one of the most complex elements of the CMIP6
infrastructure due to the complexity of the new design outlined in
\cite{ref:eyringetal2016a}. The experimental design now involves 3
tiers of experiments, where an individual modelling group may choose
which ones to perform; and variables grouped by scientific goals and
priorities, where again centres may choose which sets to publish,
based on interests and resource constraints. There are also
cross-experiment data requests, where for instance the design may
require a variable in one experiment to be compared against the same
variable from a different experiment. The modelling groups will then
need to take this into account before beginning their simulations. The
CMIP6 Data Request is a codification of the entire experimental design
into a structured set of machine-readable documents, which can in
principle be directly ingested in data workflows.

The \href{http://clipc-services.ceda.ac.uk/dreq/index.html}{CMIP6 Data
  Request}\footnote{http://clipc-services.ceda.ac.uk/dreq/index.html,
  retrieved \today.} \citep{ref:juckesetal2015} combines definitions
of variables and their output format with specifications of the
objectives they support and the experiments that they are required
for. The entire request is encoded in an XML database with rigorous
type constraints. Important elements of the request, such as units,
cell methods (expressing the subgrid processing implicit in the
variable definition), sampling frequencies, and time ``slices''
(subsets of the entire simulation period as defined in the
experimental design) for required output, are defined using controlled
vocabularies that ensure consistency of interpretation. The request is
designed to enable flexibility, allowing modelling centres to make
informed decisions about the variables they should submit to the CMIP6
archive from each experiment.

% The data request spans several elements.

% \begin{enumerate}
% \item specification of the parameter to be calculated in terms of a CF
%   standard name and units,
% \item an output frequency,
% \item a structural specification which includes specification of
%   dimensions and of subgrid processing.
% \end{enumerate}

In order to facilitate the cross linking between the 2100 variables
from the 287 experiments, the request database allows MIPs to
aggregate variables and experiments into groups. This allows MIPs to
designate variable groups by priority and provides for queries that
return the list of variables needed from any given experiment at a
specified time slice and frequency.
% The link between variables and
% experiments is then made through the following chain:

% \begin{itemize}
% \item A \emph{variable group}, aggregating variables with priorities
%   specific to the MIP defining the group;
% \item A \emph{request link} associating a variable group with an
%   objective and a set of request items;
% \item \emph{Request} items associating a particular time slice with a
%   request link and a set of experiments.
% \end{itemize}

This formulation takes into account the complexities that arise when a
particular MIP requests that variables needed for their own
experiments should also be saved from a DECK experiment or from an
experiment proposed by a different MIP.

The data request supports a broad range of users who are provided with
a range of different access points. These include the entire
codification in the form of a structured (XML) document, web pages, or
spreadsheets, as well as a python API and command-line tools, to
satisfy a wide variety of usage patterns for accessing the data
request information.

% \begin{enumerate}
% \item The XML database provides the reference document;
% \item Web pages provide a direct representation of the database
%   content;
% \item Excel workbooks provide selected overviews for specific MIPs and
%   experiments;
% \item A python library provides an interface to the database with some
%   built-in support functions;
% \item A command line tool based on the python library allows quick
%   access to simple queries.
% \end{enumerate}

The data request's machine-readable database has been an extraordinary
resource for the modelling centres. They can, for example, directly
integrate the request specifications with their workflows to ensure
that the correct set of variables are saved for each experiment they
plan to run. In addition, it has given them a new-found ability to
estimate the data volume associated with meeting a MIP's requirements,
a feature exploited below in Section~\ref{sec:dvol}.

\subsection{Model inputs}
\label{sec:data-inputs}

Datasets used by the model for configuration of model inputs
\citep[\texttt{Input Datasets for Model Intercomparison Projects)
  input4MIPs}, see][]{ref:duracketal2018} as well as observations for
comparison with models \citep[\texttt{Observations for Model
  Intercomparison Projects) obs4MIPs},
see][]{ref:teixeiraetal2014,ref:ferraroetal2015} are both now
organised in the same way, and share many of the naming and metadata
conventions as the CMIP model output itself. The coherence of
standards across model inputs, outputs, and observational datasets is
a development that will enable the community to build a rich toolset
across all of these datasets. The datasets follow the versioning
methodologies described in Section~\ref{sec:version}.

\subsection{Data Reference Syntax}
\label{sec:data-drs}

The organisation of the model output follows the
\href{https://docs.google.com/document/d/1h0r8RZr_f3-8egBMMh7aqLwy3snpD6_MrDz1q8n5XUk/edit?usp=sharing
}{Data Reference Syntax
  (DRS)}\footnote{https://docs.google.com/document/d/1h0r8RZr\_f3-8egBMMh7aqLwy3snpD6\_MrDz1q8n5XUk/edit?usp=sharing
  , retrieved \today.} first used in CMIP5, and now in a somewhat
modified form in CMIP6. The DRS depends on pre-defined
\emph{controlled vocabularies}
\href{https://github.com/WCRP-CMIP/CMIP6_CVs}{CMIP6\_CVs}\footnote{https://github.com/WCRP-CMIP/CMIP6\_CVs,
  retrieved \today.} for various terms including: the names of
institutions, models, experiments, time frequencies, etc. The CVs are
now recorded as a version-controlled set of structured text documents,
and satisfies the requirement that there is a
\href{https://github.com/WCRP-CMIP/CMIP6_CVs }{single authoritative
  source for any CV}\footnote{https://github.com/WCRP-CMIP/CMIP6\_CVs
  , retrieved \today.} , on which all elements in the toolchain will
rely. The DRS elements that rely on these controlled vocabularies
appear as netCDF attributes and are used in constructing file names,
directory names, and unique identifiers of datasets that are essential
throughout the CMIP6 infrastructure. These aspects are covered in
detail in the
\href{https://www.earthsystemcog.org/site_media/projects/wip/CMIP6_global_attributes_filenames_CVs_v6.2.6.pdf
}{CMIP6 Global Attributes, DRS, Filenames, Directory Structure, and
  CVs}\footnote{https://www.earthsystemcog.org/site\_media/projects/wip/CMIP6\_global\_attributes\_filenames\_CVs\_v6.2.6.pdf
  , retrieved \today.} position paper. A new element in the DRS
indicates whether data have been stored on a native grid or have been
regridded (see discussion below in Section~\ref{sec:dvol} on the
potentially critical role of regridded output). This element of the
DRS will allow us to track the usage of the \emph{regridded subset} of
data and assess the relative popularity of native-grid vs.
standard-grid output.

\subsection{CMIP6 data volumes}
\label{sec:dvol}

As noted, extrapolations based on CMIP3 and CMIP5 lead to some
alarming trends in data volume \citep[see
e.g.,][]{ref:overpecketal2011}. As seen in their Figure~2, model
output such as those from the various CMIP phases (1 through 6) are
beginning to rival observational data volume. As noted in the
introduction, a particular problem for our community is the diverse
and very large user base for the data, many of whom are not climate
specialists, but downstream users of climate data studying the impacts
of climate change. This stands in contrast to other fields with
comparably large data holdings: data from the Large Hadron Collider
\citep[e.g.,][]{ref:aadetal2008}, for example, is primarily consumed
by high energy physicists and not of direct interest to anyone else.

A rigorous approach is needed to estimate future data volumes, rather
than relying on simple extrapolation. Contributions to the increase in
data volume include the systematic increase in model resolution and
complexity of the experimental protocol and data request. We consider
these separately:

\begin{description}
\item[Resolution] The median horizontal resolution of a CMIP model
  tends to grow with time, and is expected to be more typically 100~km
  in CMIP6, compared to 200~km in CMIP5. Typically the temporal
  resolution of the model (though not the data) is doubled as well,
  for reasons of numerical stability. Thus, for an $N$-fold increase
  in horizontal resolution, we require an $N^3$ increase in
  computational capacity. The vertical resolution grows in a more
  controlled fashion, at least as far as the data is concerned, as
  often the requested output is reported on a standard set of
  atmospheric levels that has not changed much over the years.
  Similarly the temporal resolution of the data request does not
  increase at the same rate as the model timestep: monthly averages
  remain monthly averages. Thus, the $N^3$ increase in computational
  capacity will result in an $N^2$ increase in data volume,
  \emph{ceteris paribus}. Thus, data volume $V$ and computational
  capacity $C$ are related as $V \sim C^\frac23$, purely from the
  point of view of resolution. Consequently, if centres then
  experience an 8-fold increase in $C$ between CMIPs, we can expect a
  doubling of model resolution and an approximate quadrupling of the
  data volume (see discussion in the
  \href{https://docs.google.com/document/d/1kZw3KXvhRAJdBrXHhXo4f6PDl_NzrFre1UfWGHISPz4/edit?ts=5995cbff
  }{CMIP6 Output Grid Guidance
    document}\footnote{https://docs.google.com/document/d/1kZw3KXvhRAJdBrXHhXo4f6PDl\_NzrFre1UfWGHISPz4/edit?ts=5995cbff
    , retrieved \today.} ).

  A similar approximate doubling of model resolution occurred between
  CMIP3 and CMIP5, but data volume increased 50-fold. What caused that
  extraordinary increase?
\item[Complexity] The answer lies in the complexity of CMIP: the
  complexity of the data request and of the experimental protocol. The
  first component, the data request complexity, is related to that of
  the science: the number of processes being studied, and the physical
  variables required for the study, along with the large number of
  satellite MIPs (23) that now comprise the CMIP6 project. In CPMIP
  \citep{ref:balajietal2017}, we have attempted a rigorous definition
  of this complexity, measured by the number of physical variables
  simulated by the model. This, we argue, grows not smoothly like
  resolution, but in very distinct generational step transitions, such
  as the one from atmosphere-ocean models to Earth system models,
  which, as shown in \cite{ref:balajietal2017}, involved a substantial
  jump in complexity with regard to the number of physical, chemical,
  and biological species being modelled. Many models of the CMIP5 era
  added atmospheric chemistry and aerosol-cloud feedbacks, sometimes
  with $\mathcal{O}(100)$ species. CMIP5 also marked the first time in
  CMIP that ESMs were used to simulate changes in the carbon cycle.

  % the following increase in complexity doesn't help explain the 50-fold increase 
  % which is what this paragraph is supposed to address
  %  the number of experiments (or number of years simulated) are
  % primarily controlled by $C$, which you say is limited to 8-fold increase.
  %  need to restructure the argument.
  The second component of complexity is the experimental protocol, and
  the number of experiments themselves when comparing successive
  phases of CMIP. The number of experiments (and years simulated) grew
  from 12 in CMIP3 to about 50 in CMIP5, greatly inflating the data
  produced. With the new structure of CMIP6, with a DECK and 23
  endorsed MIPs, the number of experiments has grown tremendously
  (from about 50 to 287). We propose as a measure of experimental
  complexity, the \emph{total number of simulated years (SYs)} called
  for by the experimental protocol. Note that modelling centres must
  make tradeoffs between experimental complexity and resolution in
  deciding their level of participation in CMIP6, as discussed in
  \cite{ref:balajietal2017}.
\end{description}

Two further steps have been proposed toward ensuring sustainable
growth in data volumes.
% Given the earlier arguments, it seems $C$ will limit growth of volume by itself
%  Why are additional steps necessary?
The first of these is the consideration of standard horizontal
resolutions for saving data, as is already done for vertical and
temporal resolution in the data request. Cross-model analyses already
cast all data to a common grid in order to evaluate it as an ensemble,
typically at fairly low resolution. The studies of Knutti and
colleagues (e.g., \cite{ref:knuttietal2017}), for example, are
typically performed on relatively coarse grids. Accordingly for most
purposes atmospheric data on the ERA-40 grid
($2^\circ\times 2.5^\circ$) would suffice, with obvious exceptions for
experiments like those called for by HighResMIP
\citep{ref:haarsmaetal2016}. A similar conclusion applies for ocean
data (the World Ocean Atlas $1^\circ\times 1^\circ$ grid), with
extended discussion of the benefits and losses due to regridding
\citep[see][]{ref:griffiesetal2014,ref:griffiesetal2016}.

This has not been mandated for CMIP6 for a number of reasons. Firstly,
regridding is burdensome on many grounds: It requires considerable
expertise to choose appropriate algorithms for particular variables,
for instance, we may need ones that guarantee exact conservation for
scalars or preservation of streamlines for vector fields may be a
requirement; and it can be expensive in terms of computation and
storage. Secondly, regridding is irreversible (thus amounting to
``lossy'' data reduction) and non-commutative with certain basic
arithmetic operations such as multiplication (i.e., the product of
regridded variables does not in general equal the regridded output of
the product computed on the native grid). This can be problematic for
budget studies. However, the same issues would apply for
time-averaging and other operations long used in the field: much
analysis of CMIP output is performed on monthly-averaged data, which
is ``lossy'' compression along the time axis relative to the model's
time resolution.

These issues have contributed to a lack of consensus in moving
forward, and the recommendations on regridding remain in flux. The
\href{https://docs.google.com/document/d/1kZw3KXvhRAJdBrXHhXo4f6PDl_NzrFre1UfWGHISPz4/edit?ts=5995cbff
}{CMIP6 Output Grid Guidance
  document}\footnote{https://docs.google.com/document/d/1kZw3KXvhRAJdBrXHhXo4f6PDl\_NzrFre1UfWGHISPz4/edit?ts=5995cbff
  , retrieved \today.} outlines a number of possible recommendations,
including the provision of ``weights'' to a target grid. Many of the
considerations around regridding, particularly for ocean data in
CMIP6, are discussed at length in \cite{ref:griffiesetal2016}.

There is a similar lack of consensus around whether or not to adopt a
common \emph{calendar} for particular experiments. In cases such as a
long-running control simulation where all years are equivalent and of
no historical significance, it is customary in this community to use
simplified calendars -- such as a Julian, a ``noleap'' (365-day) or
``equal-month'' (360-day) calendar -- rather than the Gregorian.
However, comparison across datasets using different calendars can be a
frustrating burden on the end-user. There is no consensus at this
point, however, to impose a particular calendar.

As outlined below in Section~\ref{sec:replica}, both ESGF data nodes
and the creators of secondary repositories are given considerable
leeway in choosing data subsets for replication, based on their own
interests. The tracking mechanisms outlined in Section~\ref{sec:pid}
below will allow us to ascertain, after the fact, how widely used the
native grid data may be \emph{vis-\`a-vis} the regridded subset, and
allow us to recalibrate the replicas, as usage data becomes available.
We note also that the providers of at least one of the standard
metrics packages \citep[ESMValTool,][]{ref:eyringetal2016a} have
expressed a preference of standard grid data for their analysis, as
regridding from disparate grids increases the complexity of their
already overburdened infrastructure.

A second method of data reduction for the purposes of storage and
transmission is the issue of data compression. The netCDF4 software,
which is used in writing CMIP6 data, includes an option for lossless
compression or deflation \citep{ref:zivlempel1977} that relies on the
same technique used in standard tools such as \texttt{gzip}. In
practice, the reduction in data volume will depend upon the
``entropy'' or randomness in the data, with smoother data or fields
with many missing data points (e.g. land or ocean) being compressed
more.

Dealing with compressed data entails computational costs, not only
during its creation, but also every time the data are re-inflated.
There is also a subtle interplay with precision: for instance
temperatures usually seen in climate models appear to deflate better
when expressed in Kelvin, rather than Celsius, but that is due to the
fact that the leading order bits are always the same, and thus the
data is actually less precise. Deflation is also enhanced by
reorganising (``shuffling'') the data internally into chunks that have
spatial and temporal coherence.

Some argue for the use of more aggressive \emph{lossy} compression
methods \citep{ref:bakeretal2016}, but for CMIP6 it can be argued that
the resulting loss of precision and the consequences for scientific
results require considerably more evaluation by the community before
such methods can be accepted. However, as noted above, some lossy
methods of data reduction (e.g., time-averaging) have long been common
practice.

To help inform the discussion about compression, we undertook a
systematic study of typical model output files under lossless
compression, the results of which are
\href{https://public.tableau.com/profile/balticbirch\#!/vizhome/NC4/NetCDF4Deflation}{publicly
  available}\footnote{https://public.tableau.com/profile/balticbirch\#!/vizhome/NC4/NetCDF4Deflation,
  retrieved \today.} . The study indicates that standard \texttt{zlib}
compression in the netCDF4 library with the settings of
\texttt{deflate=2} (relatively modest, and computationally
inexpensive), and \texttt{shuffle} (which ensures better
spatiotemporal homogeneity) ensures the best compromise between
increased computational cost and reduced data volume. For an ESM, we
expect a total savings of about 50\%, with ocean, ice, and land realms
benefiting most (owing to large areas of the globe that are masked)
and atmospheric data benefiting least. This 50\% estimate has been
verified with sample output from one model whose compression rates
should be quite typical.

The \href{https://earthsystemcog.org/projects/wip/CMIP6DataRequest
}{DREQ}\footnote{https://earthsystemcog.org/projects/wip/CMIP6DataRequest
  , retrieved \today.} alluded to above in Section~\ref{sec:dreq}
allows us to estimate expected data volumes. The software generates an
estimate given the model's resolution along with the experiments that
will be performed and the data one intends to save (using DREQ's
\emph{priority} attribute).
% With this information
% We are actually capturing this information in the registered content
% for the model source_id entries - see http://rawgit.com/WCRP-CMIP/CMIP6_CVs/master/src/CMIP6_source_id.html
% The json entry contains resolutions for each active model realm
% https://github.com/WCRP-CMIP/CMIP6_CVs/blob/master/CMIP6_source_id.json
%  "unprecedented" is incorrect.
% In CMIP5 we had a sophisticated capability of estimating data volume
%  We polled the groups to determine which experiments they planned
% to run and how large their ensembles would be.  
%  We also asked what resolution they would report output.
%  From this we estimated in Nov. 2010 a total data volume of 2.5 petabytes 
%  (2.1 petabytes if only high-priority variables were reported), not too 
% far from the actual volume.  I'll send you the analysis if you like.
% The modelling groups had access to this information.
% !PD! IS THE BELOW UP TO DATE? yes
For instance, analyses available at the
\href{http://clipc-services.ceda.ac.uk/dreq/tab01_3_3.html}{DREQ
  site}\footnote{http://clipc-services.ceda.ac.uk/dreq/tab01\_3\_3.html,
  retrieved \today.} indicate that if a centre were to undertake every
single experiment (all tiers) and save every single variable requested
(all priorities) at a ``typical'' resolution, it would generate about
800~TB of data, using the guidelines above. Given 100 participating
models, this translates to an upper bound of 80~PB for the entire
CMIP6 archive, though in practice most centres are planning to perform
a modest subset of experiments and save only a subset of variables,
based on their scientific priorities and available computational and
storage resources. The WIP carried out a survey of modelling centres
in 2016, asking them for their expected model resolutions, and
intentions of participating in various experiments. Based on that
survey, we initially have forecast a compressed data volume of 18~PB
for CMIP6. This number, 18~PB, is about 7 times the CMIP5 archive
size. The causes for this dramatic increase in data volume between
CMIP3 and CMIP5 were noted above. There is no comparable jump between
CMIP5 and CMIP6. CMIP6's innovative DECK/endorsed-MIP structure could
be considered successful in that it has limited the rate of growth in
data volume.

Prior to CMIP5, similar analyses were undertaken at PCMDI to estimate
data volume and the predicted volume proved reasonably accurate. The
methods used for CMIP5, however, could not be applied to CMIP6 because
they depended on having a much less complex data request. In
particular, the cross-MIP data requests (variables requested by one
MIP from another MIP, or the DECK) require a more sophisticated
algorithm. The experience in many modelling centres as present is that
data volume estimates become available only after the production runs
have begun. Reliable estimates \emph{ahead of time} based on nothing
more than the experimental protocols and model resolutions are
valuable for preparation and planning hardware acquisitions.

% if you want to discuss different grids, perhaps here is a better
% place for that.
It should be noted that reporting output on a lower resolution
standard grid (rather than the native model grid) could shrink the
estimated data volume 10-fold, to 1.8~PB. This is an important number,
as will be seen below in Section~\ref{sec:replica}: the managers of
Tier~1 nodes (the largest nodes in the federation) have indicated that
2~PB is about the practical limit for replicated storage of data from
all CMIP6 models.
% I for one don't think it is important for all the data to be replicated
This target could be achieved by requiring compression and the use of 
reduced-resolution standard grids, but modelling centres are free to choose
whether or not to compress and regrid.

\section{Licensing}
\label{sec:licensing}

The licensing policy established for CMIP6 is based on an examination
of data usage patterns in CMIP5. First, while in CMIP5 the licensing
policy called for registration and acceptance of the terms of use, a
large fraction, perhaps a majority of users, actually obtained their
data not directly from ESGF, but from third-party copies, such as the
``snapshots'' alluded to in Item~\ref{snap},
Section~\ref{sec:principles}. Those users accessing the data
indirectly, as shown in Figure~\ref{fig:dark}, relied on user groups
or their home institutions to make secondary repositories that could
be more conveniently accessed. The WIP
\href{https://www.earthsystemcog.org/site_media/projects/wip/CMIP6_Licensing_and_Access_Control.pdf
}{CMIP6 Licensing and Access
  Control}\footnote{https://www.earthsystemcog.org/site\_media/projects/wip/CMIP6\_Licensing\_and\_Access\_Control.pdf
  , retrieved \today.} position paper refers to the secondary
repositories as ``dark'' and those obtaining CMIP data from those
repositories as ``dark users'' who are invisible to the ESGF system.
While this appears to subvert the licensing and registration policy
put in place for CMIP5, this should not be seen as a ``bootleg''
process: it is in fact the most efficient use of limited network
bandwidth and storage at the user sites. In CMIP6 we expect similar
data archive snapshots to host data and offload some of the network
provisioning requirements from the ESGF nodes.

At the same time we wish to retain the ability for users of these
``dark'' repositories to benefit from the augmented provenance
services provided by infrastructure advances, where a user can inform
themselves or be notified of data retractions or replacements when
contributed datasets are found to be erroneous and replaced (see
Section~\ref{sec:cite} and Section~\ref{sec:doc}).

\begin{figure*}
  \begin{center}
    \includegraphics[width=175mm]{images/WIP-data-process.png}
  \end{center}
  \caption{Typical data access pattern in CMIP5 involved users making
    local copies, and user groups making institutional-scale caches
    from ESGF. Figure courtesy Stephan Kindermann, DKRZ, adapted from
    WIP Licensing White Paper.}
  \label{fig:dark}
\end{figure*}

The proposed licensing policy removes the impossible task of license
enforcement from the distribution system, and embraces the ``dark''
repositories and users. To quote the WIP position paper:

\begin{quote}
  The proposal is that (1) a data license be embedded in the data
  files, making it impossible for users to avoid having a copy of the
  license, and (2) the onus on defending the provisions of the license
  be on the original modeling center...
\end{quote}

Licenses will be embedded in all CMIP6 files, and all repositories,
whether sanctioned or ``dark'', can be data sources, as seen below in
the discussion of replication (Section~\ref{sec:replica}). In the
embedded license approach, modelling centres are offered two choices
of \emph{Creative Commons} licenses: data covered by the
\href{http://creativecommons.org/licenses/by-sa/4.0/ }{Creative
  Commons Attribution ``Share Alike'' 4.0 International
  License}\footnote{http://creativecommons.org/licenses/by-sa/4.0/ ,
  retrieved \today.} will be freely available; for centres with more
restrictive policies, the
\href{http://creativecommons.org/licenses/by-nc-sa/4.0/ }{Creative
  Commons Attribution ``NonCommercial Share Alike'' 4.0 International
  License}\footnote{http://creativecommons.org/licenses/by-nc-sa/4.0/
  , retrieved \today.} will limit use to non-commercial purposes.
Further sharing of the data is allowed, as the license travels with
the data. The PCMDI website provides a link to the current
\href{https://pcmdi.llnl.gov/CMIP6/TermsOfUse}{CMIP6 Terms of Use
  webpage}\footnote{https://pcmdi.llnl.gov/CMIP6/TermsOfUse, retrieved
  \today.} .

\section{Citation, provenance, quality assurance, and documentation}
\label{sec:cite}

As noted in Section~\ref{sec:principles}, citation requirements flow
from two underlying considerations: one, to provide proper credit and
formal acknowledgment of the authors of datasets; and the other, to
enable rigorous tracking of data provenance and data usage. The
tracking facilitates scientific reproducibility and traceability, as
well as enabling statistical analyses of dataset utility.

In addition to clearly identifying what data have been used in
research studies and who deserves credit for providing that data, it
is essential that the data be examined for quality and that
documentation be made available describing the model and experiment
conditions under which it was generated. These subjects are addressed
in the four position papers summarized in this section.

The principles outlined above are well-aligned with the
\href{https://www.force11.org/group/joint-declaration-data-citation-principles-final
}{Joint Declaration of Data Citation
  Principles}\footnote{https://www.force11.org/group/joint-declaration-data-citation-principles-final
  , retrieved \today.} formulated by the Force11 (The Future of
Research Communications and e-Scholarship) Consortium, which has
acknowledged the rapid evolution of digital scholarship and archival,
as well as the need to update the rules of scholarly publication for
the digital age. We are convinced that not only peer-reviewed
publications but also the data itself should now be considered a
first-class product of the research enterprise. This means that data
requires curation and should be treated with the same care as journal
articles. Moreover, most journals and academies now insist that data
used in the literature be made publicly available for independent
inquiry and reproduction of results. New services like
\href{http://www.scholix.org}{Scholix}\footnote{http://www.scholix.org,
  retrieved \today.} are evolving to support the exchange and access
of such data-data and data-literature interlinking.

Given the complexity of the CMIP6 data request, we expect a total
dataset count of $\mathcal{O}(10^6)$. Because dozens of datasets are
typically used in a single scientific study, it is impractical to cite
each dataset individually in the same way as individual research
publications are acknowledged. Based on this consideration, there
needs to be a mechanism to cite data and give credit to data providers
that relies on a rather coarse granularity, while at the same time
offering another option at a much finer granularity for recording the
specific files and datasets used in a study.

In the following, two distinct types of persistent identifiers (PIDs)
are discussed: DOIs, which can only be assigned to data that comply
with certain standards for citation metadata and curation, and the
more generic
\href{https://www.dona.net/handle-system}{``Handles''}\footnote{https://www.dona.net/handle-system,
  retrieved \today.} that have fewer constraints and may be more
easily adapted for a particular use. The Handle system, as explained
in Section~\ref{sec:pid} allows unique PIDs to be assigned to datasets
at the point of publication. Technically both types of PIDs rely on
the underlying global Handle System to provide services (e.g., to
resolve the PIDs and provide associated metadata, such as the location
of the data itself).

\subsection{Persistent identifiers for acknowledgment and citation}
\label{sec:doi}

Based on earlier phases of CMIP, some datasets contributed to the
CMIP6 archive will be flawed (due, for example, to errors in
processing) and therefore will not accurately represent a model's
behavior. When errors are uncovered in the datasets, they may be
replaced with corrected versions. Similarly, additional datasets may
be added to an initially incomplete collection of datasets. Thus,
initially at least, the DOIs assigned for the purposes of citation and
acknowledgement will represent an evolving underlying collection of
datasets.

The recommendations, detailed in the
\href{https://www.earthsystemcog.org/site_media/projects/wip/CMIP6_Data_Citation_LTA.pdf
}{CMIP6 Data Citation and Long Term
  Archival}\footnote{https://www.earthsystemcog.org/site\_media/projects/wip/CMIP6\_Data\_Citation\_LTA.pdf
  , retrieved \today.} position paper, recognize two phases to the
process of assigning DOI's to collections of datasets: an initial
phase, when the data have been released and preliminary community
analysis is underway and a second stage when most errors in the data
have been identified and corrected. Upon reaching stage two, the data
will be transferred to long-term archival (LTA) of the IPCC Data
Distribution Centre (IPCC DDC) and deemed appropriate for
interdisciplinary use (e.g., in policy studies).

For evolving dataset aggregations, the data citation infrastructure
relies on information collected from the data providers and uses the
\href{https://www.datacite.org/dois.html}{DataCite}\footnote{https://www.datacite.org/dois.html,
  retrieved \today.} data infrastructure to assign DOIs and record
associated metadata. DataCite is a leading global non-profit
organisation that provides persistent identifiers (DOIs) for research
data. The DOIs will be assigned to:

\begin{enumerate}
\item aggregations that include all the datasets contributed by one
  model from one institution from all of a single MIP's experiments,
  and
\item smaller-size aggregations that include all datasets contributed
  by one model from one institution generated in performing one
  experiment (which might include one or more simulations).
\end{enumerate}

These aggregations are dynamic as far as the PID infrastructure is
concerned: new elements can be added to the aggregation without
modifying the PID. As an example, for the coarser of the two
aggregations defined above, the same PID will apply to an evolving
number of simulations as new experiments are performed with the model.
This PID architecture is shown in Figure~\ref{fig:pidarch}. Since
these collections are dynamic, citation requires authors to provide a
version reference.

\begin{figure*}
  \begin{center}
    \includegraphics[width=175mm]{images/PID-architecture.png}
  \end{center}
  \caption{Schematic PID architecture, showing layers in the PID
    hierarchy. In the lower layers of the hierarchy, PIDs are static
    once generated, and new datasets generate new versions with new
    PIDs. Each file carries a PID and each collection (dataset,
    simulation, ..) is related to a PID. Resolving the PID in the
    Handle server guides the user to the file or the landing page
    describing the collection. Each box in the figure will be
    addressed uniquely by its PID.}
  \label{fig:pidarch}
\end{figure*}

As an initial dataset matures and becomes stable, it is assigned a new
DOI. Before this is done, to meet formal requirements, the data
citation infrastructure requires some additional steps. First, we
ensure that there has been sufficient community examination of the
data (through citations in published literature, for instance) to
qualify it as having been peer-reviewed. Second, further steps are
undertaken to assure important information exists in ancillary
metadata repositories, including, for example, documentation (ES-DOC,
errata and citation) and to provide quality assurance of data and
metadata consistency and completeness (see Section~\ref{sec:qa}). Once
these criteria have been satisfied, a DOI will be issued by the IPCC
DDC hosted by DKRZ. These dataset collections will meet the stringent
metadata and documentation requirements of the IPCC DDC. Since these
collections are static, no version reference is required in a
citation. Should errors be found subsequently, they will be corrected
in the data and published under a new DOI. The original DOI and its
related data are still available but are labeled as superseded with a
link recorded pointing to the corrected data.

For CMIP6, the initially assigned DOIs (associated with evolving
collections of data) must be used in research papers to properly give
credit to each of the modelling groups providing the data. Once a
stable collection of datasets has met the higher standards for
long-term curation and quality, the DOI assigned by the IPCC DDC
should be used instead. The data citation approach is described in
greater detail in \cite{ref:stockhauselautenschlager2017}.

\subsection{Persistent identifiers for tracking, provenance, and
  curation}
\label{sec:pid}

Although the DOIs assigned to relatively large aggregations of
datasets are well suited for citation and acknowledgment purposes,
they are not issued at fine enough granularity to meet the scientific
imperative that published results should be traceable and verifiable.
Furthermore, management of the CMIP6 archive requires that PIDs be
assigned at a much finer granularity than the DOIs. For these
purposes, PIDs recognized by the global Handle registry will be
assigned at two different levels of granularity: one per file and one
per dataset.

A unique Handle will be generated each time a new CMIP6 data file is
created, and the Handle will be recorded in the file's metadata (in
the form of a netCDF global attribute named \texttt{tracking\_id}). At
the time the data is published, the \texttt{tracking\_id} will be
processed by the CMIP6 Handle service infrastructure and recorded in
the ESGF metadata catalog. Another Handle will subsequently be
assigned at somewhat coarser granularity to each aggregation of files
containing the data from a single variable sampled at a single
frequency from a single model running a single experiment. In ESGF
terminology, this collection of files is referred to as an
\emph{atomic dataset}.

As described in the
\href{https://www.earthsystemcog.org/site_media/projects/wip/CMIP6_PID_Implementation_Plan.pdf
}{CMIP6 Persistent Identifiers Implementation
  Plan}\footnote{https://www.earthsystemcog.org/site\_media/projects/wip/CMIP6\_PID\_Implementation\_Plan.pdf
  , retrieved \today.} position paper, a Handle assigned at either of
these two levels of the PID hierarchy identifies a static entity; if
any file associated with a Handle is altered in any way a new Handle
must be created. The PID infrastructure is also central to the
replication and versioning strategies, as described in
Section~\ref{sec:replica} and Section~\ref{sec:version} below.
Furthermore, as a means of recording provenance and enabling tracking
of dataset usage, authors are urged to include as supplementary
material attached to each CMIP6-based publication a PID list (a flat
list of all PIDs referenced).

\begin{figure*}
  \begin{center}
    \includegraphics[width=175mm]{images/PID-workflow.png}
  \end{center}
  \caption{PID workflow, showing the generation and registry of PIDs,
    with checkpoints where compliance is assured.}
  \label{fig:pidflow}
\end{figure*}

The implementation plan describes methods for generating and
registering Handles using an asynchronous messaging system known as
RabbitMQ. This system, designed in collaboration with ESGF developers
and shown in Figure~\ref{fig:pidflow}, guarantees, for example, that
PIDs are correctly generated in accordance with the versioning
guidelines. The CMIP6 handle system builds on the idea of tracking-ids
used in CMIP5, but with a more rigorous quality control to ensure that
new PIDs are generated when data are modified. The dataset and file
Handles are also associated with basic metadata, called PID kernel
information \citep{ref:zhouetal2018}, which facilitate the recording
of basic provenance information. Datasets and files point to each
other to bind the granularities together. In addition, dataset kernel
information refers to previous and later versions, errata information
and replicas, as explained in more detail in the position paper.

\subsection{Quality Assurance}
\label{sec:qa}

Quality assurance (QA) encompasses the entire data lifecycle, as
depicted in Figure~\ref{fig:qa}. At all stages, a goal is to capture
provenance information that will enable scientific reproducibility.
Further, as noted in Item~\ref{broad} in Section~\ref{sec:principles},
the QA procedures should uncover issues that might undermine trust in
the data by those outside the Earth system modelling community if
errors were left unreported.

\begin{figure*}
  \begin{center}
    \includegraphics[width=175mm]{images/WIP-QA.png}
  \end{center}
  \caption{Schematic of the phases of quality assurance, with earlier
    stages in the hands of modelling centres (left), and more formal
    long-term data curation stages at right. Quality assurance is
    applied both to the data (D, above) as well as the metadata (M)
    describing the data. Figure drawn from the WIP's Quality Assurance
    position paper.}
  \label{fig:qa}
\end{figure*}

QA must ensure that the data and metadata correctly reflect a model's
simulation, so that it can be reliably used for scientific purposes.
As depicted in Figure~\ref{fig:qa}, the first stage of QA is the
responsibility of the data producer: in fact the cycle of model
development and diagnosis is the most critical element of QA. The
second aspect is ensuring that disseminated data include common
metadata based on common CVs, which will enable consistent treatment
of data from different groups and institutions. These requirements are
directly embedded in the ESGF publishing process and in tools such as
\href{https://cmor.llnl.gov/}{CMOR}\footnote{https://cmor.llnl.gov/,
  retrieved \today.} (and its validation component,
\href{https://cmor.llnl.gov/mydoc_cmip6_validator/
}{PrePARE}\footnote{https://cmor.llnl.gov/mydoc\_cmip6\_validator/ ,
  retrieved \today.} ). These checks (the D1 and M1 phases of QA in
Figure~\ref{fig:qa}) ensure that the data conform to the CMIP6 Data
Request specifications, conform to all naming conventions and CVs, and
follow the mandated structure for organisation into a common directory
structure. As noted in Section~\ref{sec:dreq}, many modelling centres
have chosen to embed these steps directly in their workflows to ensure
conformance with the CMIP6 requirements as the models are being run
and their output processed.

At this point, as noted in Figure~\ref{fig:qa}, control is ceded to
the ESGF system, where designated QA nodes (ESGF data nodes where
additional services are turned on) perform further QA checks to
certify data is suitable for citation and long-term archival). A
critical step is the assignment of PIDs (Section~\ref{sec:pid}, the D2
stage of Figure~\ref{fig:pidflow}), which is more controlled than in
CMIP5 and guarantees that across the data lifecycle, the PIDs will be
reliably useful as unique labels of datasets.

Beyond this, further stages of QA will be handled within the ESGF
system following procedures outlined in the
\href{https://www.earthsystemcog.org/site_media/projects/wip/CMIP6_Quality_Assurance.pdf
}{CMIP6 Quality
  Assurance}\footnote{https://www.earthsystemcog.org/site\_media/projects/wip/CMIP6\_Quality\_Assurance.pdf
  , retrieved \today.} position paper. As described before, once data
have been published, the data will be scrutinized by researchers in
what can be considered an ongoing period of community-wide scientific
QA of the data. During this period, modelling centres may correct
errors and provide new versions of datasets. In the final stage, the
data pass into long term archival (LTA) status, described as the
``bibliometric'' phase in Figure~\ref{fig:qa}. Just prior to LTA, the
system will verify minimum standards of provenance documentation. This
is described in the next section.

\subsection{Documentation of provenance}
\label{sec:doc}

As noted earlier in Section~\ref{sec:dreq}, for data to become a
first-class scientific resource, the methods of their production must
be documented to the fullest extent possible. For CMIP6, this includes
documenting both the models and the experiments. While traditionally
this is done through peer-reviewed literature, which remains
essential, we note that to facilitate various aspects of search,
discovery and tracking of datasets, there is an additional need for
structured documentation in machine readable form.

\begin{figure*}
  \begin{center}
    \includegraphics[width=120mm]{images/cmip6_workflow_infographic_v2.pdf}
  \end{center}
  \caption{Elements of ES-DOC documentation. Rows indicate phases of
    the modelling process being documented, and box colors indicate the
    parties responsible for producing the documentation (see legend).
    Figure courtesy Guillaume Levavasseur, IPSL}.
  \label{fig:esdoc}
\end{figure*}

In CMIP6, the documentation of \emph{experiments}, \emph{models} and
\emph{simulations} is done through the Earth System Documentation
\citep[\href{https://www.earthsystemcog.org/projects/es-doc-models/
}{ES-DOC}\footnote{https://www.earthsystemcog.org/projects/es-doc-models/
  , retrieved \today.} ,][]{ref:guilyardietal2013} Project. The
various aspects of model documentation are shown in
Figure~\ref{fig:esdoc}, and in greater detail in the WIP position
paper on
\href{https://www.earthsystemcog.org/site_media/projects/wip/CMIP6_ESDOC_documentation.pdf
}{ES-DOC}\footnote{https://www.earthsystemcog.org/site\_media/projects/wip/CMIP6\_ESDOC\_documentation.pdf
  , retrieved \today.} . The CMIP6 experimental design has been
translated into structured text documents, already available from
ES-DOC. ES-DOC has constructed CVs for the description of the CMIP6
standard model realms (CMIP terminology for climate subsystems, such
as ``ocean'' or ``atmosphere''), including a set of short tables
(\emph{specialisations}, in ES-DOC terminology) for each realm. The
specialisations are a succinct and structured description of the model
physics. Ideally, modelling groups would integrate with their model
development process their provision of documentation to ES-DOC. This
would better ensure the accuracy and consistency of the documentation.
ES-DOC provides a variety of user interfaces to read and write
structured documentation that conforms with the Common Information
Model (CIM) of \cite{ref:lawrenceetal2012}. As models evolve or
differentiate (for example, an Earth system model derived from a
particular physics-only general circulation model), branches and new
versions of the documentation can be produced, and it will be possible
to display, annotate, and add new entries in the genealogy of a model
in a manner familiar to anyone who works with version control software
like \texttt{git}.

A critical element in the ES-DOC process is the documentation of
\emph{conformances}: steps undertaken by the modelling centres to
ensure that the simulation was conducted as called for by the
experiment design. It is here that that the input datasets used in a
simulation are documented \citep[e.g., the version of each of the
forcing datasets, see][]{ref:duracketal2018}. The conformances will be
an important element in guiding selection of subsets of CMIP6 model
results for particular research studies. A researcher might, for
example, choose to subselect only those models that used a particular
version of the forcing datasets that are imposed as part of the
experimental protocol. The conformances will continue to grow in
importance under the CMIP vision that the DECK will provide an ongoing
foundation on which to build a series of future CMIP phases
\citep[shown schematically in Figure~1 of][]{ref:eyringetal2016a}. The
conformances will be essential in enabling studies across model
generations.

The method of capturing the conformance documentation is a two-stage
process that has been designed to minimize the amount of work required
by a modelling centre. The first stage is to capture the many
conformances common to all simulations. ES-DOC will then automatically
copy these common conformances to multiple simulations thereby
eliminating duplicated effort. This is followed by a second stage in
which those conformances that are specific to individual experiments
or simulations are collected.

While this method of documentation is unfamiliar to many, such methods
are likely to become common and required practice in the maturing
digital age as part of best scientific practices. Documentation of
software validation \citep[see e.g][]{ref:peng2011} and structured
documentation of complete scientific workflows that can be
independently read and processed are both becoming more common
\citep[see the special issue on the ``Geoscience Paper of the
Future'', ][]{ref:davidetal2016}. We have noted earlier (see
Item~\ref{repro} in Section~\ref{sec:principles}) the special
importance in climate research today of documenting how results have
been obtained and enabling results to be reproduced by others.
Rigorous documentation remains a hardy bulwark against challenges to
the scientific process.

In keeping with the ``dataset-centric rather than system-centric''
approach (Item~\ref{snap} in Section~\ref{sec:principles}), a user
will be directly linked to documentation from each dataset. This is
done in CMIP6 by adding a required global attribute
\texttt{further\_info\_url} in file headers pointing to the associated
CIM document, which will serve as the landing page for documentation
from which further exploration (by humans or software) will take
place. The form of this URL is standard and can be software-generated:
CMOR, for instance, will automatically add it. The existence and
functioning of the landing page is assured in Stage~M3 of
Figure~\ref{fig:qa}.

\section{Replication}
\label{sec:replica}

The replication strategy is covered in the
\href{https://www.earthsystemcog.org/site_media/projects/wip/CMIP6_Replication_and_Versioning.pdf
}{CMIP6 Replication and
  Versioning}\footnote{https://www.earthsystemcog.org/site\_media/projects/wip/CMIP6\_Replication\_and\_Versioning.pdf
  , retrieved \today.} position paper. The recommendations therein are
based on the following \emph{primary} goal:

\begin{itemize}
\item Ensuring at least one copy of a dataset is present at a stable
  ESGF node with a mission of long-term maintenance and curation of
  data. The total data storage resources planned across the Tier~1
  nodes in the CMIP6 era is adequate to support this requirement,
  though some data will likely be held on accessible tape storage
  rather than spinning disk.
\end{itemize}

In addition, we have articulated a number of secondary goals:

\begin{itemize}
\item Enhancing data accessibility across the ESGF (e.g. Australian
  data easily accessible to the European continent despite the long
  distance);
\item Enabling each Tier 1 data node to enact specific policies to
  support their local objectives;
\item Ensuring that the most widely requested data is accessible from
  multiple ESGF data nodes; (of course, any dataset will be available
  at least on its original publication datanode);
\item Enabling large-scale data analysis across the federation (see
  Item~\ref{analysis} in Section~\ref{sec:principles});
\item Ensuring continuity of data access in the event of individual
  node failures;
\item Enabling network load-balancing and enhanced performance;
\item Reducing the manual workload related to replication;
\item Building a reliable replication mechanism that can be used not
  only within the federation, but by the secondary repositories
  created by user groups (see discussion in
  Section~\ref{sec:licensing} around Figure~\ref{fig:dark}).
\end{itemize}

In conjunction with the ESGF and the International Climate Networking
Working Group (ICNWG), these recommendations have been translated to
two options for replication.

The basic toolchain for replication is built on updated versions of
the software layers used in CMIP5 including:
\href{https://github.com/Prodiguer/synda}{synda}\footnote{https://github.com/Prodiguer/synda,
  retrieved \today.} (formerly \texttt{synchrodata}) and Globus Online
\citep{ref:chardetal2015}, which are based on underlying data
transport mechanisms such as
\href{http://toolkit.globus.org/toolkit/docs/latest-stable/gridftp/
}{gridftp}\footnote{http://toolkit.globus.org/toolkit/docs/latest-stable/gridftp/
  , retrieved \today.} and the older and now deprecated protocols like
\texttt{wget} and \texttt{ftp}.

As one option, these layers can be used for \emph{ad hoc} replication
by sites or user groups. For \emph{ad hoc} replication, there is no
obvious mechanism for triggering updates or replication when new or
corrected data are published (or retracted, see
Section~\ref{sec:version} below). As a second option, certain
designated nodes (\emph{replica nodes}) will maintain a protocol for
automatic replication, shown in Figure~\ref{fig:replica}.

\begin{figure*}
  \begin{center}
    \includegraphics[width=120mm]{images/WIP-replication.png}
  \end{center}
  \caption{CMIP6 replication from data nodes to replica centres and
    between replica centres coordinated by a CMIP6 replication team,
    under the guidance of the CDNOT.}
  \label{fig:replica}
\end{figure*}

Given the nature of some of the secondary goals listed above, it would
not be appropriate to prescribe which data should be replicated by
each centre. Rather, the plan should be flexible to accommodate
changing data use profiles and resource availability. A replication
team under the guidance of the CDNOT will coordinate the replication
activities of the CMIP6 data nodes such that the primary goal is
achieved and an effective compromise for the secondary goals is
established.

The International Climate Network Working Group (ICNWG), formed under
the Earth System Grid Federation (ESGF), helps set up and optimize
network infrastructures for ESGF climate data sites located around the
world. For example, prioritising the most widely requested data for
replication can best be done based on operational experience and will
of course change over time. To ensure that the replication strategy is
responding to user need and data node capabilities, the replication
team will maintain and run a set of monitoring and notification tools
assuring that replicas are up-to-date. The CDNOT is tasked with
ensuring the deployment and smooth functioning of replica nodes.

A key issue that emerged from discussions with node managers is that
the replication target has to be of sustainable size. A key finding is
that a replication target about 2~PB in size is the practical
(technical and financial) limit for CMIP6 online (disk) storage at any
single location. Replication beyond this may involve offline storage
(tape) for disaster recovery.

Based on experience in CMIP5, it is expected that a number of
``special interest'' secondary repositories will hold selected subsets
of CMIP6 data outside of the ESGF federation. This will have the
effect of widening data accessibility geographically, and by user
communities, with obvious benefit to the CMIP6 project. These
secondary repositories will be encouraged and supported where it does
not undermine CMIP6 data management and integrity objectives.

In the new dataset-centric approach, licenses and PIDs remain embedded
and will continue to play their roles in the data toolchain even for
these secondary repositories.

In CMIP5 a significant issue for users of some third-party archives
was that their replicated data was taken as a one-time snapshot (see
discussion above in Item~\ref{snap} in Section~\ref{sec:principles}),
and not updated as new versions of the data were submitted to the
source ESGF node. Tools have been developed by a number of
organisations to maintain locally synchronized archives of CMIP5 data
and third party providers should be encouraged to make use of these
types of tools to keep the local archives up to date.

In summary, the requirements for replication are limited to ensuring:

\begin{itemize}
\item that within a reasonably short time period following submission,
  there is at least one instance of each submitted dataset stored at a
  Tier~1 node (in addition to its primary residence);
\item that subsequent versions of submitted datasets are also
  replicated by at least one Tier~1 node (see versioning discussion
  below in Section~\ref{sec:version});
\item that creators of secondary repositories take advantage of the
  replication toolchain described here, to maintain replicas that can
  be kept up to date, and inform local users of dataset retractions
  and corrections;
\item that the CDNOT is the recognized body to manage the operational
  replication strategy for CMIP6.
\end{itemize}

We note that the the ESGF PID registration service is part of the ESGF
data publication implementation and not exclusive to CMIP6, and is now
in use by the input4MIPs and obs4MIPs projects. The PID registration
service works for all NetCDF-CF files that carry a PID as
\texttt{tracking\_id} field. This is agreed for all CMIP6 data files.
However, the ESGF PID registration service is not exclusively
applicable for CMIP6 model data files but can also be used for derived
data sets (e.g., subsets or averages) as long as the data are in
NetCDF-CF format with a PID from the Handle service in the
\texttt{tracking\_id}. Once the data are processed by the ESGF PID
registration service, these files may easily be easily be used to
create collections in the PID hierarchy as given in
Figure~\ref{fig:pidarch}. In general all files as digital objects can
be assigned a PID and registered in the CNRI Handle server. Vice
versa, these objects (files) can be uniquely resolved by the Handle
server providing the PID is known. That means the PID service allows
for stable and transparent data access independently from the actual
storage location. The storage location is part of the PID meta data
which are integrated in the in the Handle server. The PID metadata
generation and registration is part of the ESGF registration service
for NetCDF-CF files but in general the PID architecture is not
restricted to them. It is open for all digital objects.

Thus, CMIP6 is the first implementation of the PID service in a larger
data project and ESGF provides in parallel the classical data access
via the Data Reference Syntax outlined in the
\href{https://www.earthsystemcog.org/site_media/projects/wip/CMIP6_global_attributes_filenames_CVs_v6.2.6.pdf
}{CMIP6 Global Attributes, DRS, Filenames, Directory Structure, and
  CVs}\footnote{https://www.earthsystemcog.org/site\_media/projects/wip/CMIP6\_global\_attributes\_filenames\_CVs\_v6.2.6.pdf
  , retrieved \today.} position paper.

\section{Versioning}
\label{sec:version}

The versioning strategy for CMIP6 datasets (see the
\href{https://www.earthsystemcog.org/site_media/projects/wip/CMIP6_Replication_and_Versioning.pdf
}{CMIP6 Replication and
  Versioning}\footnote{https://www.earthsystemcog.org/site\_media/projects/wip/CMIP6\_Replication\_and\_Versioning.pdf
  , retrieved \today.} position paper) is designed to enable
reproduction of scientific results (Section~\ref{sec:principles}).
Recognizing that errors may be found after datasets have been
distributed, erroneous datasets that may have been used downstream
will continue to be publicly available but marked as superseded. This
will allow users to trace the provenance of published results even if
those point to retracted data and will further allow the possibility
of \emph{a~posteriori} correction of such results.

A consistent versioning methodology across all the ESGF data nodes is
required to satisfy these objectives. We note that inconsistent or
informal versioning practices at individual nodes would likely be
invisible to the ESGF infrastructure (e.g., yielding files that look
like replicas, but with inconsistent data and checksums), which would
inhibit traceability across versions.

Building on the replication strategy and on input from the ESGF
implementation teams, versioning will leverage the PID infrastructure
of Section~\ref{sec:cite}. PIDs are permanently associated with a
dataset, and new versions will get a new PID. When new versions are
published, there will be two-way links created within the PID kernel
information so that one may query a PID for prior or subsequent
versions.

A version number will be assigned to each \emph{atomic dataset}: a
complete timeseries of one variable from one experiment and one model.
The implication is that if an error is found in a single variable,
other variables produced from the simulation need not be republished.
If an entire experiment is retracted and republished, all variables
will get a consistent version number. The CDNOT will ensure consistent
versioning practices at all participating data nodes.

\subsection{Errata}
\label{sec:errata}

% The following description of CMIP5 errata is not quite right and
% should be revised.
It is worth highlighting in particular the new recommendations
regarding errata. Until CMIP5, we have relied on the ESGF system to
push notifications to registered users regarding retractions and
reported errors. This was found to result in imperfect coverage: as
noted in Section~\ref{sec:licensing}, a substantial fraction of users
are invisible to the ESGF system. Therefore, following the discussion
in Section~\ref{sec:principles} (see Item~\ref{snap}), we have
recommended a design which is dataset-centric rather than
system-centric. Notifications are no longer pushed to users; rather
they will be able to query the status of a dataset they are working
with (e.g.
\href{https://errata.es-doc.org/static/index.html}{ES-DOC Dataset
  Errata search}\footnote{https://errata.es-doc.org/static/index.html,
  retrieved \today.} ).
An \emph{errata client} will allow the user to enter a PID to
query its status; and an \emph{errata server} will return the PIDs
associated with prior or posterior versions of that dataset, if any.
Details are to be found in the \href{https://www.earthsystemcog.org/site_media/projects/wip/CMIP6_Errata_System.pdf
}{Errata}\footnote{https://www.earthsystemcog.org/site\_media/projects/wip/CMIP6\_Errata\_System.pdf
, retrieved \today.} 
position paper.

\conclusions[The future of the global data infrastructure]
\label{sec:summary}

The WIP was formed in response to the explosive growth of CMIP between
CMIP3 and CMIP5, and it is charged with studying and making
recommendations about the global data infrastructure needed to support
CMIP6 and subsequent similar WCRP activities as they are established
and evolve. Our findings reflect the fact that CMIP is no longer a
cottage industry, and a more formal approach is needed. Several of the
findings have been translated into requirements on the design of the
underlying software infrastructure for data production and
distribution. We have separated infrastructure development into
requirements, implementation, and operations phases, and we have
provided recommendations on the most efficient use of scarce
resources. The resulting recommendations stop well short of any sort
of global governance of this ``vast machine'', but address many areas
where, with a relatively light touch, beneficial order, control, and
resource efficiencies result.

One key finding that informs everything is that it appears that the
critical importance of such infrastructure is under-appreciated.
Building infrastructure using research funds puts the system in an
untenable position, with a fundamental contradiction at its heart:
infrastructure by its nature should be reliable, robust, based on what
is proven to work, and invisible, whereas scientific research is
hypothesis-driven, risky and novel, and its results widely broadcast.
While recommendations have been made at the highest level advocating
remedies \citep[e.g.,][]{ref:nasem2012}, there is little progress on
this front to report. Several of the key pieces of infrastructure
software described here are built and tested by volunteers or
short-term project staff.

The central theme of this paper is the inversion of the design of
federated data distribution, to make it \emph{dataset-centric rather
  than system-centric}. We believe that this one aspect of the design
considerably reduces systemic risk, and allows the size of the system
to scale up and down as resource constraints allow. Individual
scientists or institutions or consortia, will be able to pool
resources and share data at will, with relatively light requirements
related to licensing (Section~\ref{sec:licensing}) and dataset
tracking (Section~\ref{sec:pid}). This relieves a considerable design
burden from the ESGF software stack, and further, recognizes that the
data ecosystem extends well beyond the reach of any software system
and that data will be used and reused in a myriad of ways outside
anyone's control.

A second key element of the design is the insistence on
\emph{machine-readable experimental protocols}. Standards,
conventions, and vocabularies are now stored in machine-readable
structured text formats like XML and JSON, thereby enabling software
to automate aspects of the process. This meets an existing urgent
need, with some modelling centres already exploiting this structured
information to mitigate against the overwhelming complexity of
experimental protocols. Moreover, this will also enable and encourage
unanticipated future use of the information in developing new software
tools for exploiting it as technologies evolve. Our ability to predict
(whether correctly or not remains to be seen) the expected CMIP6 data
volume is one such unexpected outcome.

Finally, the infrastructure allows user communities to assess the
\emph{costs of participation} as well as the benefits. For example, we
believe the new PID-based methods of dataset tracking will allow
centres to measure which data has value downstream. The importance of
citations and fair credit for data providers is recognized with a
design that facilitates and encourages proper citation practices.
Tools have been added and made available that allow centres, and the
CMIP itself, to estimate data requirements of each experimental
protocol. Ancillary activities such as CPMIP add to this an accounting
of the computational burden of CMIP6.

Certainly not all issues are resolved, and the validation of some of
our findings will have to await the outcome of CMIP6. There is no
community consensus on some proposed design elements, such as standard
grids. Some features long promised, such as server-side analytics
(``bringing analysis to the data'') are yet to mature, however the
ESGF community is actively working on this through the ESGF Compute
Working Team (CWT, see
\href{https://esgf.llnl.gov/media/2017-F2F/Day2/Day2-CWT_Presentation.pdf}{ESGF
  2017 Face to Face Compute Working Team
  update}\footnote{https://esgf.llnl.gov/media/2017-F2F/Day2/Day2-CWT\_Presentation.pdf,
  retrieved \today.}), although many exciting efforts are underway,
for instance early investigations at using cloud technologies, both
for data storage and analysis. Nevertheless, the discussion in this
article provides a sound basis for beginning to think about the
future.

The future brings with it new challenges. First among these is an
expansion of the data ecosystem. There is an increasing blurring of
the boundary between weather and climate as time and space scales
merge \citep{ref:hoskins2013}. This will increasingly entrain new
communities into climate data ecosystems, each with their own
modelling and analysis practices, standards and conventions, and other
issues. The establishment of the WIP was a crucial step in enhancing
the capabilities, standards, protocols and policies around the CMIP
enterprise. Earlier discussions on the scope of the WIP also suggested
a broader scope for the panel on the longer-term, to coordinate not
only the model intercomparison activities (including for example, the
CORDEX project \citep{ref:lakeetal2017}, which also relies upon ESGF
for data dissemination) but also the climate prediction (seasonal to
decadal) issues and corresponding observational and reanalysis
aspects. We would recommend a closer engagement between these
communities in planning the future of a seamless global data
infrastructure, to better leverage infrastructure investments and
effort.

A further challenge the WIP and the community must grapple with is the
evolution of scientific publication in the digital age, beyond the
peer-reviewed paper. We have noted above that the nature of
publication is changing \citep[see e.g][]{ref:davidetal2016}. Journals
and academies increasingly insist upon transparency with respect to
codes and data to ensure reproducibility. In the future, datasets and
software with provenance information will be first-class entities of
scientific publication, alongside the traditional peer-reviewed
article. In fact it is likely that those will increasingly be featured
in the grey literature and scientific social media: one can imagine
blog posts and direct annotations on the published literature around
CMIP6 using analysis directly performed on datasets using their PIDs.
Data analytics at large scale is increasingly moving toward machine
learning and other directly data-driven methods of analysis, which
will also be dependent on data labelled with machine-readable
metadata. Our community needs to pay increasing heed to the status of
their data, metadata, and software in the light of these developments.

Future development of the WIP's activities beyond the delivery of
CMIP6 will include an analysis of how the infrastructure design
performed during CMIP6. That analysis, combined with our assessment of
technological change and emerging novel applications, will inform
future design of infrastructure software, as well as recommendations
to the designers of experiments on how best to fit their protocols
within resource limitations. The vision, as always, is for an open
infrastructure that is reliable and invisible, and allows Earth system
scientists to be nimble in the design of collaborative experiments,
creative in their analysis, and rapid in the delivery of results.

\appendix

\section{List of WIP position papers}
\label{sec:wip}


\begin{itemize}
\item
  \href{https://www.earthsystemcog.org/site_media/projects/wip/CDNOT_Terms_of_Reference.pdf
  }{CDNOT Terms of
    Reference}\footnote{https://www.earthsystemcog.org/site\_media/projects/wip/CDNOT\_Terms\_of\_Reference.pdf
    , retrieved \today.} : a charter for the CMIP6 Data Node
  Operations Team. Authorship: WIP.
\item
  \href{https://www.earthsystemcog.org/site_media/projects/wip/CMIP6_global_attributes_filenames_CVs_v6.2.6.pdf
  }{CMIP6 Global Attributes, DRS, Filenames, Directory Structure, and
    CVs}\footnote{https://www.earthsystemcog.org/site\_media/projects/wip/CMIP6\_global\_attributes\_filenames\_CVs\_v6.2.6.pdf
    , retrieved \today.} : conventions and controlled vocabularies for
  consistent naming of files and variables. Authorship: Karl E.
  Taylor, Martin Juckes, V. Balaji, Luca Cinquini, Sébastien Denvil,
  Paul J. Durack, Mark Elkington, Eric Guilyardi, Slava Kharin,
  Michael Lautenschlager, Bryan Lawrence, Denis Nadeau, and Martina
  Stockhause, and the WIP.
\item
  \href{https://www.earthsystemcog.org/site_media/projects/wip/CMIP6_PID_Implementation_Plan.pdf
  }{CMIP6 Persistent Identifiers Implementation
    Plan}\footnote{https://www.earthsystemcog.org/site\_media/projects/wip/CMIP6\_PID\_Implementation\_Plan.pdf
    , retrieved \today.} : a system of identifying and citing datasets
  used in studies, at a fine grain. Authorship: Tobias Weigel, Michael
  Lautenschlager, Martin Juckes and the WIP.
\item
  \href{https://www.earthsystemcog.org/site_media/projects/wip/CMIP6_Replication_and_Versioning.pdf
  }{CMIP6 Replication and
    Versioning}\footnote{https://www.earthsystemcog.org/site\_media/projects/wip/CMIP6\_Replication\_and\_Versioning.pdf
    , retrieved \today.} : a system for ensuring reliable and
  verifiable replication; tracking of dataset versions, retractions
  and errata. Authors: Stephan Kindermann, Sebastien Denvil and the
  WIP.
\item
  \href{https://www.earthsystemcog.org/site_media/projects/wip/CMIP6_Quality_Assurance.pdf
  }{CMIP6 Quality
    Assurance}\footnote{https://www.earthsystemcog.org/site\_media/projects/wip/CMIP6\_Quality\_Assurance.pdf
    , retrieved \today.} : systems for ensuring data compliance with
  rules and conventions listed above. Authorship: Frank Toussaint,
  Martina Stockhause, Michael Lautenschlager and the WIP.
\item
  \href{https://www.earthsystemcog.org/site_media/projects/wip/CMIP6_Data_Citation_LTA.pdf
  }{CMIP6 Data Citation and Long Term
    Archival}\footnote{https://www.earthsystemcog.org/site\_media/projects/wip/CMIP6\_Data\_Citation\_LTA.pdf
    , retrieved \today.} : a system for generating Document Object
  Identifies (DOIs) to ensure long-term data curation. Authorship:
  Martina Stockhause, Frank Toussaint, Michael Lautenschlager, Bryan
  Lawrence and the WIP.
\item
  \href{https://www.earthsystemcog.org/site_media/projects/wip/CMIP6_Licensing_and_Access_Control.pdf
  }{CMIP6 Licensing and Access
    Control}\footnote{https://www.earthsystemcog.org/site\_media/projects/wip/CMIP6\_Licensing\_and\_Access\_Control.pdf
    , retrieved \today.} : terms of use and licenses to use data.
  Authorship: Bryan Lawrence and the WIP.
\item
  \href{https://www.earthsystemcog.org/site_media/projects/wip/CMIP6_ESGF_Publication_Requirements.pdf
  }{CMIP6 ESGF Publication
    Requirements}\footnote{https://www.earthsystemcog.org/site\_media/projects/wip/CMIP6\_ESGF\_Publication\_Requirements.pdf
    , retrieved \today.} : linking WIP specifications to the ESGF
  software stack, conventions that software developers can build
  against. Authorship: Martin Juckes and the WIP.
\item
  \href{https://www.earthsystemcog.org/site_media/projects/wip/CMIP6_Errata_System.pdf
  }{Errata System for
    CMIP6}\footnote{https://www.earthsystemcog.org/site\_media/projects/wip/CMIP6\_Errata\_System.pdf
    , retrieved \today.} : a system for tracking and discovery of
  reported errata in the CMIP6 system. Authorship: Guillaume
  Levavasseur, Sébastien Denvil, Atef Ben Nasser, and the WIP.
\item
  \href{https://www.earthsystemcog.org/site_media/projects/wip/CMIP6_ESDOC_documentation.pdf
  }{ESDOC
    Documentation}\footnote{https://www.earthsystemcog.org/site\_media/projects/wip/CMIP6\_ESDOC\_documentation.pdf
    , retrieved \today.} : An overview of the process for providing
  structured documentation of the models, experiments and simulations
  that produce the CMIP6 output datasets. Authorship: the ES-DOC Team.
\end{itemize}

\section{Data and code availability}
\label{sec:code}

\begin{itemize}
\item The software and data used for the study of data compression are
  available at
  \href{https://public.tableau.com/profile/balticbirch\#!/vizhome/NC4/NetCDF4Deflation
  }{deflation study
    website}\footnote{https://public.tableau.com/profile/balticbirch\#!/vizhome/NC4/NetCDF4Deflation,
    retrieved \today.}, courtesy Garrett Wright.
\item The software and data used for the prediction of data volumes
  are available at
  \href{https://www.earthsystemcog.org/site_media/projects/wip/dreqDataVol.py
  }{the dreqDataVol
    page}\footnote{https://www.earthsystemcog.org/site\_media/projects/wip/dreqDataVol.py,
    retrieved \today.}, courtesy Nalanda Sharadjaya. Much of this
  functionality has now been absorbed into DREQ itself.
\end{itemize}

Most of the software referenced here for which the WIP is providing
design guidelines and requirements, but not implementation, including
the ESGF, ESDOC, and ]DREQ software stacks are open source and freely
available. They are autonomous projects and therefore not listed here.

\begin{acknowledgements}
  We thank Michel Rixen, Stephen Griffies, and John Krasting for their
  close reading and comments on early drafts of this manuscript.
  Colleen McHugh aided with the analysis of data volumes.
  
  The research leading to these results has received funding from the
  European Union Seventh Framework program under the IS-ENES2 project
  (grant agreement No. 312979).

  V. Balaji is supported by the Cooperative Institute for Climate
  Science, Princeton University, Award NA08OAR4320752 from the
  National Oceanic and Atmospheric Administration, U.S. Department of
  Commerce. The statements, findings, conclusions, and recommendations
  are those of the authors and do not necessarily reflect the views of
  Princeton University, the National Oceanic and Atmospheric
  Administration, or the U.S. Department of Commerce.

  B.N. Lawrence acknowledges additional support from the UK Natural
  Environment Research Council.
  
  K.E. Taylor and P.J. Durack are supported by the Regional and Global
  Model Analysis Program of the United States Department of Energy's
  Office of Science, and their work was performed under the auspices
  of Lawrence Livermore National Laboratory's Contract
  DE-AC52-07NA27344.
\end{acknowledgements}

\bibliographystyle{copernicus}
\bibliography{refs}

\end{document}
